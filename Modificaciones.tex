\documentclass[a4paper, 12pt]{article}

\usepackage{array}
\usepackage{amssymb}
%\usepackage{times}
\usepackage[spanish, activeacute]{babel}
\usepackage{graphicx}
\usepackage{hyperref}
%\usepackage[utf8]{inputenc}
\usepackage{fancyhdr}
\usepackage{xtab}
\usepackage{color}
\usepackage{lscape}
\usepackage{longtable}
\usepackage{tabularx}
\usepackage{xltabular}
\usepackage{fontspec}
\usepackage{url}
\usepackage{colortbl}
\usepackage{graphics}

\usepackage{cite}
\usepackage{lscape}
%\usepackage[T1]{fontenc}

\usepackage{tikz}

%% -------------------------------------- Declare the layers
\pgfdeclarelayer{nodelayer}
\pgfdeclarelayer{edgelayer}
\pgfsetlayers{edgelayer,nodelayer,main}

%% -------------------------------------- Declare the styles
\tikzset{newstyle/.style={thick}}
\tikzset{simple/.style={thick}}

\tikzstyle{new style 0}=[fill={rgb,255: red,228; green,255; blue,165}, draw=black, shape=rectangle, tikzit fill={rgb,255: red,228; green,255; blue,165}, tikzit shape=rectangle, align=center]

% Edge styles
\tikzstyle{EjeColorFlecha1}=[draw=black, ->,very thick,densely dotted, fill=none, tikzit draw=black]
\tikzstyle{EjeFlechaColor2}=[draw={rgb,255: red,243; green,14; blue,37}, ->, very thick,fill=none, tikzit draw={rgb,255: red,243; green,14; blue,37}]
\tikzstyle{EjeFlechaColor3}=[draw={rgb,255: red,191; green,0; blue,64}, ->, tikzit draw={rgb,255: red,191; green,0; blue,64}]


\newenvironment{colortext}[1]{\color{#1}}{\ignorespacesafterend}
\setsansfont{Roboto Condensed}%{TeXGyrePagella}%{Linux Biolinum O}%{Junicode}%{Carlito}
\renewcommand{\familydefault}{\sfdefault}







\pagestyle{fancyplain}

 \renewcommand{\sectionmark}[1]
                 {\markright{\thesection\ #1}}

 \newcommand{\coltex}[1]{\textcolor{red}{#1}}

                 
% \lhead[\fancyplain{}{\bfseries\thepage}]
%       {\fancyplain{}{\bfseries\rightmark}}
%
 \rhead[\fancyplain{}{\bfseries\leftmark}]{\fancyplain{}{\bfseries\thepage}}


 \setlength{\headheight}{34.43594pt}

 \lhead[\fancyplain{}{\vspace{-1cm}\includegraphics[scale=.35]{membrete.png}}]{\fancyplain{}{\vspace{-1cm}\includegraphics[scale=.35]{membrete.png}}}

\cfoot{}





\hyphenation{de-ri-va-das} \hyphenation{le-bes-gue}
\hyphenation{e-llas} \hyphenation{o-cu-rrien-do}
\hyphenation{pro-pie-da-des}\hyphenation{pi-vo-te}
\hyphenation{dia-go-na-li} \hyphenation{e-cua-cion}
\hyphenation{a-pro-pia-dos}\hyphenation{ma-te-má-ti-cos}
\hyphenation{es-tu-dian-te}
\hyphenation{po-si-ti-vis-mo} \hyphenation{mo-de-li-za}

\hyphenation{gi-co}







\begin{document}
 
\section*{5.5.2  Capacidades y habilidades requeridas para la realización de las actividades que le incumben}

\textcolor{blue}{AGREGAR}

\begin{itemize}

\item Habilidad para trabajar en contextos internacionales.
 
\item Capacidad de  comprender otras formas de validación del conocimiento (metodo empírico).  

\end{itemize}

%  
%  
%  

\section*{5.6. REQUISITOS DE INGRESO} 

\textcolor{blue}{REEMPLAZAR LO QUE DICE POR LO SIGUIENTE}

Los requisitos para el ingreso a la carrera de Licenciatura en Matemática son los establecidos
en el artículo 7o de la Ley de Educación Superior. Los aspirantes deberán haber aprobado
el nivel de enseñanza secundaria. Excepcionalmente, los mayores de veinticinco años que
no reúnan esta condición podrán ingresar siempre que demuestren a través de una
evaluación que establezca nuestra Universidad, que tienen preparación y/o experiencia
laboral acorde a los estudios que se proponen iniciar, así como conocimientos y actitudes
para cursarlos satisfactoriamente.

Los aspirantes deberán además cumplir con las exigencias
 que establezcan las normativas específicas de la UNRC y de la FCEFQyN    vigentes en el momento 
 de la inscripción. 

 
\section*{5.7.3 CONTENIDOS Y METODOLOGÍA  }

\paragraph{Inglés}



\textcolor{blue}{PONER ESTOS CONTENIDOS  Y CORREGIR EN TODOS LOS LADOS QUE APAREZCA }


Inglés Intermedio (56 h)

 

Contenidos mínimos: Géneros discursivos y sus situaciones de contexto, la intencionalidad del autor y la función social del texto: boletines informativos y artículos de semi-divulgación. Léxico específico de la disciplina, estructuras léxico-gramaticales simples y complejas (densidad lexical y sintáctica). Variedad de registros, argumentación y posicionamiento del autor. Marcadores discursivos de ideas principales y secundarias.

 

Fundamentación: Se trabaja con contenidos disciplinares y lingüísticos que puedan ser transferidos a las actividades de aprendizaje que se desarrollan en las demás asignaturas que conforman el Plan de Estudio y que apoyan el desarrollo integral del estudiante.

 

Carga Horaria semanal Total: 2 h  \textbf{(ESTÁ BIEN?)}

Carga Horaria Total: 56 h

Régimen de cursado: Anual

Modalidad de enseñanza y de aprendizaje: Clases Teóricas Prácticas (56 h)

 

Inglés Avanzado (56 h)

 

Contenidos mínimos: Géneros discursivos y sus situaciones de contexto, la intencionalidad del autor y la función social del texto: el artículo de revisión bibliográfica, el artículo de investigación, el resumen, (y el estudio de caso). Léxico específico de la disciplina, estructuras léxico-gramaticales complejas a nivel lingüístico y conceptual. Representaciones multimediales de contenidos conceptuales de la disciplina y su evaluación crítica.


Fundamentación: Se trabaja con contenidos disciplinares y lingüísticos que puedan ser transferidos a las actividades de aprendizaje que se desarrollan en las demás asignaturas que conforman el Plan de Estudio y que apoyan el desarrollo integral del estudiante.

 

Carga Horaria semanal Total: 4 h

Carga Horaria Total: 56 h

Régimen de cursado: Cuatrimestral

Modalidad de enseñanza y de aprendizaje: Clases Teóricas Prácticas (56 h)

 

Pueden agregar esto, también (si quieren):

 



\paragraph{Optativas  }

\textcolor{blue}{PONER ESTE TEXTO Y LISTADO. HAY QUE FILTRAR LA LISTA DE OPTATIVAS}

Las asignaturas optativas, se eligen a partir de una nómina propuesta anualmente por Consejo Departamental de Matemática con el acuerdo de la Comisión Curricular Permanente, quienes a su vez establecen las correlatividades y son aprobadas por el Consejo Directivo de la Facultad. Los estudiantes, con acuerdo de la Comisión Curricular
Permanente, podrán a su vez cursar asignaturas avanzadas de carreras afines pertenecientes a otras carreras de la Facultad o a otras universidades nacionales internacionales, siempre que, que los contenidos mínimos y la intensidad de carga
práctica respondan a los establecidos en el presente plan y exista un convenio con la Facultad o Universidad.
Algunas de las posibles asignaturas optativas a ofrecer son:




\begin{description}

\item[Orientación A:] \emph{Análisis Matemático y Aproximación de Funciones}

\textcolor{blue}{Listado propuesto por Fabián}


\begin{enumerate}


\item\textbf{Introducción a la Teoría de Aproximación (2278):}
\emph{Contenidos sugeridos:}
Existencia, unicidad y caracterización de mejores aproximantes desde subespacios en espacios normados. Aproximación por polinomios algebraicos.  Mejor aproximación en espacios clásicos de Lebesgue. Algoritmos. Desigualdades polinomiales. Orden de aproximación. 

\noindent \textit{Bibliografía sugerida}
\cite{cheney66,lorentz2005approximation,pinkus1989l1,iske2018approximation,powell1981approximation,rice1964approximation,singer1970best}



\item\textbf{ Aproximación Simultánea en Espacios Normados (2269):}
\emph{Contenidos sugeridos:} Existencia,  unicidad y caracterización de mejores aproximantes simultáneos desde subconjuntos
convexos.  Reducción  de  mejores aproximantes simultáneos a mejores aproximantes.   Puntos  extremales  en  espacios  productos. Caracterización  funcional de  mejores aproximantes  simultáneos  Aplicaciones  a  los espacios: $C[a,b]$, $L^1[0,1]$ y $L^p[0,1]$, $1 < p <\infty$.

\noindent \textit{Bibliografía sugerida} \cite{alimov,holland,Karakus,Lin,Pinkus,Houtari,Soriano}
 

 
\item\textbf{Espacios de Funciones Invariantes por Reordenamiento:}
\emph{Contenidos sugeridos:}  Espacios de funciones de Banach. El espacio asociado. Espacios de funciones invariantes por reordenamiento. Funciones de distribución y reordenadas decrecientes. Espacios invariantes por reordenamiento. La función fundamental.  Espacios de Lorentz, $L_1+ L_\infty$ y $L_1 \cap L_\infty$. Índices de Boyd. Espacios de Orlicz-Lorentz. Teoremas de interpolación clásicos. El Teorema de Riesz-Thorin y el Teorema de Marcinkiewicz. Los espacios de Lorentz-Zygmund, $LlogL$ y $Lexp$.

\noindent \textit{Bibliografía sugerida}
\cite{bennett1988interpolation,Soria,Lindenstrauss,Krein}.



\item\textbf{Espacios de Orlicz e Interpolación:} 
\emph{Contenidos sugeridos:} Espacios modulares. Ejemplos. Espacios de Orlicz y clases de Orlicz. Separabilidad. Existencia y no existencia de funcionales lineales continuos. Función complementaria y norma de Orlicz. Forma general de funcionales lineales continuos. El producto de funciones y el Teorema de Landau. Indices es espacios de Orlicz. Espacios de Orlicz generados por funciones de Young. Teorema de interpolación no lineal de Orlicz. Interpolación en espacios de Orlicz. Espacios de Calderón-Lozanovskii e interpolación de operadores.

\noindent \textit{Bibliografía sugerida: } \cite{ Harjulehto,Krbec,KR,J. Lang,M.M. Rao}





\item \textbf{Análisis Complejo:} Principio de modulo máximo. Teorema de
Rouche. Productos infinitos. Descomposición en fracciones simples,
Teorema de Mittag-Leffer. Funciones Armónicas. Función $\zeta$ de
Riemmann. Aplicaciones a la Teoría de Números. Continuación
Analítica. Superficies de Riemann. Bibliografía sugerida:
\cite{ahlfors,conway}

\item\textbf{Complementos de Análisis Real:} Medidas abstractas. Medidas de
Radon. Teorema de Radon-Nikodyn. Derivación de Medidas. Lemas de
cubrimiento. Diferenciación de la integral. Desigualdad maximal de
Hardy y Littlewood. Teorema de diferenciación de Lebesgue.
Aproximaciones de la identidad. Aplicaciones al problema de
Dirichlet al semiespacio. Dualidad de los espacios $L^p$.
Convergencia débil. Biblio\-grafía sugerida:
\cite{evansgariepy,favazo,rudin}.


\end{enumerate}

\item[Orientación B:] \emph{Ecuaciones Diferenciales y Matemática Aplicada}
\begin{enumerate}



\item \textbf{Sistemas Dinámicos (2267)}  Flujo de una ecuación autónoma. Órbitas y conjuntos invariantes. Estabilidad en los equilibrios.   Linearización. Variedades estables e inestables. Teorema de Hartman-Grobman. Método de
Lyapunov. Estabilidad soluciones periódicas. Sistemas autónomos planos. El Teorema de Poincare Bedixson. Horas: 135. Bibliografía  sugerida: \cite{betounes2009differential,teschl2012ordinary}.

\item\textbf{ Cálculo de Variaciones (2280)}. Funciones de variación acotada y absolutamente continus. Espacios de Sobolev. El método diercto del cálculo de variaciones. Teorema de Krasnoselski. Teoremas de semicontinuidad y existencia de Tonelli. Ejemplos de no exixtencia de mínimos. Soluciones periódicas de sistemas Hamiltonianos. Horas: 154. Bibliografía  sugerida: \cite{mawhin2013critical,buttazzo1998one}.

\item\textbf{Inecuaciones Variacionales Elípticas (2279)}. Inecuaciones
Variacionales E\-lípticas en Espacios de Hilbert. Inecuaciones
Variacional con Forma Bilineal, Continua, Coercitiva y Simétrica.
Inecuación Variacional con Forma Bilineal, Continua, Coercitiva y
No Simétrica. Aplicaciones. Minimización de Funcionales en
Espacios de Banach reflexivos. Relaciones entre Inecuaciones
Variacionales y Minimización de Funciona\-les. Problemas de
Frontera Libre: Problema de la Pared Semi-permea\-ble, Problema
del Obstáculo, Fluido Viscoplástico de Bingham, Problema de Stefan
a dos fases. Análisis Numérico de inecuaciones variacionales
Elípticas. Bibliografía sugerida: \cite{ekel, kinder}


 


\item\textbf{Problemas de Frontera Libre:} Problemas de Frontera Fija, Móvil
y Libre para la Ecuación del Calor Unidimensional. Problemas de
Frontera Libre de tipo explícito e implícito. Los problemas de
Stefan y de la Difusión-Consumo de Oxígeno. Soluciones Exactas de
Lamé-Clapeyron y de Neumann, y sus Aplicaciones. Diferentes
Métodos Teóricos y Aproximados para el estudio de la solución del
problema de Stefan a una fase con condiciones de contorno de
Temperatura o Flujo de calor en el borde fijo. Comportamiento
asintótico de la Frontera Libre. El problema de Stefan a dos
fases. Bibliografía  sugerida: \cite{hill,tarzia}



\end{enumerate}

\item[Orientación C]\emph{Didáctica de la Matemática}






 


\begin{enumerate}

\item\textbf{Introducción a la Didáctica de la Matemática}
El objeto de la Didáctica de la Matemática. Distinción entre el Programa Cognitivo y el
Programa Epistemológico. Supuestos básicos y problemas iniciales del Programa
epistemológico. La Teoría de Situaciones Didácticas. \cite{Brousseau2007,Gascón2002,Gascón1998,Sadovsky2004}




\item\textbf{Desarrollos actuales de la Didáctica de la Matemática:}  La Teoría Antropológica de lo didáctico. Enfoque ontosemiótico del conocimiento y la instrucción matemáticos. Educación matemática crítica. Bibliografía sugerida: \cite{chevallard, Chevallard1999,Chevallard2001,Godino2017,Godino2007,Skovsmose1999,Skovsmose2012}.

 
\item\textbf{Epistemología de la Matemática:}
La epistemología y su relación con la historia y la filosofía de la ciencia. La epistemología como estudio de los discursos científicos: la matemática y su especificidad.  La naturaleza de los objetos y del método matemático. Las corrientes fundamentadoras clásicas: logicismo, formalismo, intuicionismo. Una perspectiva alternativa: el análisis histórico-crítico de la geometría y el álgebra.  
. Bibliografía sugerida: \cite{
klimo2,Barker1965,Gascon2001,Piaget1986,Pooper1956}




\end{enumerate}





\item[Orientación D]\emph{Estadística}
\begin{enumerate}



\item \textbf{Inferencia Estadística}.
Estimación puntual. Métodos de Estimación.
Evaluación de Estimadores.  Estimación por
Intervalos. Test de Hipótesis.

 

Referencias: \cite{bergero, bickel, degroot, lehmann,sen, rohatgi, scher}.


\item \textbf{Modelos Lineales}. Modelos de regresión. Análisis de la varianza. Distribución de formas cuadráticas y lineales. Estimación e inferencia para modelos lineales. 
 
 

 Referencias:  \cite{hock,  radrao, rawlyngs, scheffe, searle,  Venables, Montgonery}.
 
 
  
 

\item \textbf{Estadística computacional}. 
Optimización   continua y combinatoria. Algoritmos de ascensos por coordenadas. Simulated annealing. Algortimo EM. Simulación. Cadena de Markov Monte Carlo. Algoritmo Metropolis Hastings. Muestreo de Gibbs. Bootstrap. Bootstrap robusto.  


Referencias:    \cite{wrma, wrre, giho}




\item \textbf{Procesos Estocásticos}. Teorema de extensión de  medidas de Kolmogorov. Construcción de procesos a partir
de las distribuciones finito-dimensionales. Teorema de  clase $\pi-\lambda$. Esperanza condicional. Martingalas a tiempo
discreto. Desigualdades fundamentales. Teoremas de convergencia. Cadenas de Markov en espacio de estados discretos.
Clasificación de estados. Medidas invariantes. Teoría ergódica. Transformaciones que preservan medida. Teorema ergódico.

 

 Referencias: \cite{bremaud,ferrari, shir,   varadhan}.


\end{enumerate}


\item[Orientación E]\emph{Geometría, Álgebra y Grupos de Lie}.

 \begin{enumerate}
\item\textbf{ Grupos y Álgebras de Lie (3368):}  Grupos de Lie. Álgebras de Lie. La Representación Adjunta y la forma de Killing. Álgebras de Lie nilpotentes y solubles.
Horas: 126. Bibliografía sugerida: \cite{boothby2003introduction, helgason2001differential}.

\item\textbf{ Variedades Diferenciables y Riemannianas  (2277):}
Formas diferenciales. Variedades diferenciales. Integración
sobre variedades. Variedades Riemannianas. Tensor de curvatura.  Ecuaciones estructurales.
 Horas: 144. Bibliografía: \cite{boothby2003introduction,do2012differential,spivak1988cálculo}

\item\textbf{ Aproximación a la Teoría de Galois (2274):} Ampliaciones algebraicas y trascendentes de un cuerpo.  Cuerpo raíz de una ecuación. El grupo de Galois. Irresolubilidad de la ecuación de quinto grado. Horas: 135. Bibliografía sugerida: \cite{herstein,artin}.
\end{enumerate}



\end{description}


\paragraph{Metodología} \mbox{}\\


\textcolor{blue}{SE SACA TODO Y SE PONE:}


\paragraph{Actividades asignaturas de matemática}\mbox{}\\

 

\begin{description}

\item{ \textbf{Clases Expositivas.}} Se expondrán los conocimientos  que constituyen el   
fundamento teórico de las distintas materias. Se promoverá la participación  y el análisis crítico de los conocimientos impartidos, de modo tal de desarrollar en el estudiante su habilidad en transmitir ideas en lenguaje matemático y de construir y desarrollar argumentaciones lógicas con una identificación clara
de hipótesis y conclusiones.



\item{\textbf{Clases de problemas.}} En estas clases el/la estudiante es expuesto a situaciones problemáticas en el contexto de los conocimientos propios de la   materia o de su aplicación a otro   contexto. Se procurará que el/la estudiante desarrolle estrategias de resolución de  problemas de manera autónoma, que adquiera dominio del lenguaje de la matemática y del proceso de validación del conocimiento en esta ciencia. 

\item{\textbf{Clases de laboratorio de computación.}} En algunas materias, particularmente en  \emph{ Taller de informática, Cálculo
 Numérico Computacional, Ecuaciones Diferencia,   Modelos de Regresión y Métodos Empíricos e Introducción a las Ecuaciones en Derivadas
Parciales}, se destinarán horarios de práctica con computadora donde el estudiante deberá desarrollar programas que resuelvan problemas computacionales que se le presente. Se estimulará que el estudiante adquiera habilidades para analizar grandes conjuntos de datos, contribuir en la construcción de modelos matemáticos a partir de situaciones reales, utilizar las herramientas computacionales de cálculo numérico y simbólico para resolver problemas. Se espera conseguir destreza en el manejo de algunos lenguajes: m (Octave-Matlab), R, Latex y Python. 





\item{\textbf{Lectura y Escritura.}} Con el fin de <<Promover la enseñanza y el aprendizaje de prácticas de lectura y escritura crítica...que potencien competencias comunicativas y cognitivas para el desempeño de diversas prácticas universitarias.>> \footnote{Convocatoria VI PELPA, UNRC}  se incorporará  como estrategia de enseñanza de las distintas  materias la lectura (individual o grupal) de materiales teóricos y su posterior comunicación (oral o escrita). Se propenderá a la incorporación  de prácticas de enseñanza exitosas en el terreno de la lectura y escritura desarrolladas en la UNRC \cite{roldan2022} y a la participación de programas y proyectos destinados a estos fines (PELPA).  Paralelamente, la lectura de materiales en otras lenguas, particularmente inglesa, consolidará la capacidad de los estudiantes para leer y comprender textos en idioma extranjero.  


\item{\textbf{Cursos online y webinars.}} Con el propósito de fortalecer  la participación en procesos de internacionalización de la educación\footnote{Plan estratégido 2019 FECFQyN}  se  incorporara en algunas asignaturas como actividad complementaria el uso de materiales escritos y/o audiovisuales producidos por centros de reconocido prestigio para las materias que se desarrollan en ellos. Por ejemplo, las plataformas \href{https://www.edx.org/es}{https://www.edx.org/es}, \href{https://ocw.mit.edu/}{https://ocw.mit.edu/} y  \href{https://cassyni.com/c/springer-math}{https://cassyni.com/c/springer-math} comparten cursos de diversas instituciones, Massachusetts Institute of Technology y Harvard University entre ellas, y ofrece, en muchos casos de manera libre, los materiales de estos cursos, incluídas video-clases. Estas actividades contribuiran a mejorar la capacidad del estudiante de leer, escribir y comunicarse con
otros especialistas en idioma inglés. 


\item{\textbf{Disertantes invitados.}} Se alentará la participación en las materias de disertantes invitados. Estas exposiciones pueden tener carácter presencial o virtual.   

\item{\textbf{Trabajo Final.}}   Con la asistencia de su tutor el/la estudiante deberá leer los antecedentes bibliográficos que están directamente relacionados con el tema de su trabajo. Se procurará que esta bibliografía incluya artículos de investigación en revistas de reconocido prestigio, de modo tal de contribuir a la capacidad del estudiante para aprender, actualizarse y trabajar de manera autónoma y además  comprender las formas de transmitir conocimientos nuevos en las ciencias. Luego el/la estudiante  redactará la monografía, la cual puede contener resultados teóricos ya conocidos o se pueden presentar resultados originales.


\item{\textbf{Prácticas Socio-comunitarias.}}De acuerdo
con la Res. CS N° 322/2009 las mismas tienen el objeto de ``construir y afianzar un currículo que coadyuve a la creación de conciencia social y
ciudadana, en el marco de una función crítica de la Universidad''. Se prevé la inclusió de  prácticas socio-comunitarias  dentro de las
asignaturas Matemática y Sociedad, Ecuaciones Diferenciales y Modelos de Regresión y Métodos Empíricos.
\end{description}


\paragraph{Asignatura Física (1930)}\mbox{}\\

Comprende clases teóricas, clases prácticas  de resolución de problemas y clases prácticas de laboratorio. Se destinará la misma carga horaria  para cada una de estas clases. Los contenidos de la asignatura están referidos a conocimientos básicos y generales sobre las Leyes y Teorías más elementales de la Física, como también un abordaje especial sobre su metodología. Se pretende con ello proporcionar al futuro graduado el soporte necesario, en lo que a esta ciencia se refiere, para afrontar temas de su especialidad en dónde la Física tiene protagonismo. Se adaptará el desarrollo de los contenidos teóricos y prácticos a cuestiones propias de la matemática, aplicando conceptos elementales del Cálculo como análisis de gráficas para determinar trayectorias, movimientos, rotaciones, etc. Aplicaciones del concepto de límite, derivadas, integrales indefinidas y definidas para expresar magnitudes físicas en distintas situaciones. Se realizarán  mediciones científicas en el laboratorio, como soporte al análisis teórico de fenómenos físicos para lograr una visión más amplia e integrada entre la Física y la Matemática.   Las clases de laboratorio contribuiran a desarrollar en el estudiantes la capacidad de  comprender otras formas de validación del conocimiento (metodo empírico). También se estimulará su habilidad para comprender otros paradigmas y lenguajes científicos y para trabajar en equipos interdisciplinarios.
 


\paragraph{Asignaturas Humanísticas (Inglés, Sociología de la Educación)}\mbox{}\\

En las asignaturas de idioma inglés  se trabaja con contenidos disciplinares y lingüísticos que puedan ser transferidos a las actividades de aprendizaje que se desarrollan en las demás asignaturas que conforman el Plan de Estudio y que apoyan el desarrollo integral del estudiante.

La inclusión de la asignatura Sociología de la Educación se fundamenta en la necesidad de brindar al futuro egresado/a con categorías de conceptuales y marcos teóricos que le permiten analizar críticamente los procesos de enseñanza. 

\section*{5.7.4 Contenidos transversales}

\begin{description}
  



\item{\textbf{Competencias}.} Proponer al/ a la estudiante prácticas y actividades que desarrollen las competencias enumeradas en la subsección \ref{subsec:competencias}. Identificar de manera crítica cuáles de estas competencias son desarrolladas en cada actividad curricular.


\item{\textbf{Aplicaciones}.}  Contextualizar el conocimiento matemático dentro de la ciencia en general y dentro de la sociedad. Desarrollar aplicaciones de los conocimientos a otras áreas del saber y a la resolución de problemas del mundo real. 



\item{\textbf{Articulaciones verticales y horizontales.}} Coordinar el abordaje de  temas entre los distintos espacios curriculares. Establecer estrategias conjuntas para la enseñanza y asociaciones entre materias relacionadas, ya sea por ser contemporáneas en el dictado, por correlatividad o por pertenecer a un mismo trayecto del plan. 


\item{ \textbf{Sentido de los saberes.}} <<Problematizar, indagar y reflexionar constantemente el sentido de la formación
universitaria con la intención de mejorarla; además de asumir la responsabilidad social del
conocimiento en términos de su integración con las prácticas profesionales, de investigación, de
vinculación con el contexto y sustentada en principios éticos y de transformación hacia una
sociedad justa y con valores igualitarios, sustentables y ciudadanos.>> \footnote{Resolución CS-UNRC 297/2017, ``Hacia   un   currículo contextualizado, flexible e integrado. Lineamientos para la orientación de la innovación  curricular''}

\item{\textbf{Trabajo en equipo}.} Incentivar en el/la estudiante estrategias de trabajo colectivo. Proponer actividades curriculares   de carácter  grupal.



\item{ \textbf{Alfabetización académica.}} Proponer a los/las estudiantes actividades que los preparen para la
elaboración de sus trabajos finales de grado, para la investigación y para la escritura en general. 




\item{\textbf{Prácticas Socio-Comunitarias.} } Incorporar a las materias este tipo de prácticas.


\item{ \textbf{Integración de tecnologías de la información}.} Proponer a los/las estudiantes actividades que incluyan el uso de las tecnologías de la información y la programación.  



\item{\textbf{Políticas de género.}}
En el contexto de diferentes transformaciones institucionales como la Creación de la Comisión de Género de la Unión Matemática Argentina, la incorporación del apartado 6.f) al Artículo Segundo del nuevo Estatuto de la UMA, la creación de la Red Federal de Género y Diversidades del CONICET, entre otros antecedentes, se plantea la importancia de acompañar estas transformaciones con  modificaciones curriculares. 
En este plan y en consonancia con la incorporación del apartado 6.f) al Artículo Segundo del nuevo Estatuto de la UMA, se  busca promover la equidad en relación a los derechos de las mujeres e identidades disidentes en todos los quehaceres matemáticos  como así también,  procurar la eliminación de todo tipo de violencia y discriminación basadas en la identidad sexo-genérica.
También, se busca favorecer “las acciones de discriminación positiva que tiendan a superar los problemas relacionados con las inequidades y el no reconocimiento de derechos”  y    “ promover el apoyo de las vocaciones matemáticas en niñas y adolescentes y jóvenes. Visibilizar las posibilidades y los logros de matemáticas como modo de promoción de vocaciones”.
 
 \item{\textbf{Documentación}} Para otras precisiones sobre criterios para la implementación de este plan se sugiere la lectura de los lineamientos curriculares de la UNRC \footnote{``Hacia   un   currículo contextualizado, flexible e integrado'', Resolución CS-UNRC 297/2017 }
\footnote{ Resolución CS-UNRC 008/2021}, por lineamientos curriculares definidos en  \cite{paniagua2013educacion}, y por consideraciones contenidas en documentos producidos por diferentes asociaciones que agrupan profesionales matemáticos: \cite{uma,society1996siam,society2012siam,damlamian2013educational}.

\end{description}




\section*{5.7.9 Criterios}  

\textcolor{blue}{ PASAR (COMO SUGIERE LA SA) EL LISTADO QUE HAY AL PUNTO 5.4 CONTENIDOS TRANSVERSALES, DEJAR LO DEL SEGUIMIENTO AGRTEGÁNDOLE}:

\begin{itemize}

\item Entrevistando a egresados/as con el objeto de identificar conocimientos  y capacidades que den cuenta de nuevas  necesidades  emergentes en el mundo laboral.

\item Informando a los/as  estudiantes sobre convocatorias a becas de investigación para estudiantes de grado y tendiendo puentes entre ellos y los equipos de investigación de nuestro departamento. 

\item Promoviendo la participación en programas de movilidad estudiantil. 

\item Mejorando la articulación entre niveles de enseñanza. Por esto nos referimos a la consistencia entre  las competencias y conocimientos que el plan de estudios supone con aquellos   adquiridos por el estudiante en el  nivel medio de  la enseñanza. En este sentido, conjuntamente con la CCP de la Carrera de Prof. en Matemática, se propone crear un grupo de trabajo coordinado por ambas comisiones e integrado por los equipos docentes de asignaturas correspondientes a los dos primeros años de estas carreras con el fin de paulatinamente converger hacia un nuevo paradigma de enseñanza que tome en cuenta las condiciones de los/las alumnos/as ingresantes, las  miradas alternativas que surgen de la investigación en la didáctica de la matemática con enfoques no tradicionales en la enseñanza  de esta ciencia. 

\textcolor{blue}{(esto último se saca de  metodología)}

\end{itemize}


\section*{Propuesta DD.HH, prácticas socio-comunitarias}



\noindent\textbf{Matemática y sociedad} (56hs) Los siguientes contenidos se proponen como tentativos. La materia constará de dos módulos:

\begin{description}
 \item{\emph{Formación humanística.}}  Derechos humanos, género, recursos naturales, acceso a la educación, estado y sociedad. Preferentemente a cargo de un  invitado especialista en las temáticas.
 
 \item{\emph{Matemática para la justicia Social.}} El problema del recuento de votos en democracia. Teorema de imposibilidad de Arrow. Modelos matemáticos de impacto social y ambiental de la agro-industria, del transporte de hidrocarburos, incendios forestales. Teoría de grafos aplicada al estudio del tráfico de personas.  Matemática aplicada al estudio de la seguridad social. Problemas de eleccion (ejemplo niños a escuelas). Modelizando la desigualdad en los ingresos (Coeficiente de Gini). Modelos matemáticas sobre la aceptación de la diversidad sexual.   
\end{description}


\emph{Bibliografía:} \cite{buell2021mathematics,karaali2021mathematics,karaali2019mathematics,doi:10.1080/10511970.2018.1530707,doi:10.1080/10511970.2018.1512538,doi:10.1080/10511970.2018.1456498,doi:10.1080/10511970.2018.1530706,doi:10.1080/10511970.2018.1456497,doi:10.1080/10511970.2018.1472683,FredRoberts1395,JuanCarlosCortes1396,DonaldG.Saari1397,LeeRudolph1398,UrszulaStrawinska-Zanko1399,HuijiongWang1400,AhmadK.Naimzada1401,ClaudioCioffi-Revilla1402}

 


\bibliographystyle{alphaurl}
\bibliography{Bibliografia/Analisis,Bibliografia/Algebra,%
Bibliografia/Geometria,Bibliografia/BaseBibliografia,Bibliografia/Informatica,Bibliografia/Topologia,Bibliografia/TeoriaMedida,Bibliografia/probabilidad,Bibliografia/funcional,Bibliografia/modelos_sociales,Bibliografia/didactica}







\end{document}






