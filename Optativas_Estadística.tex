%==========================================================================%

\documentclass[a4paper, 12pt]{article}

\usepackage{array}
\usepackage{amssymb}
\usepackage{times}
\usepackage[spanish, activeacute]{babel}
\usepackage{graphicx}
\usepackage{hyperref}
%\usepackage[latin1]{inputenc}
\usepackage[utf8]{inputenc}
\usepackage{fancyhdr}
\usepackage{xtab}
\usepackage{color}

\hyphenation{de-ri-va-das} \hyphenation{le-bes-gue}
\hyphenation{e-llas} \hyphenation{o-cu-rrien-do}
\hyphenation{pro-pie-da-des}\hyphenation{pi-vo-te}
\hyphenation{dia-go-na-li} \hyphenation{e-cua-cion}
\hyphenation{a-pro-pia-dos}\hyphenation{ma-te-ma-ti-cos}\hyphenation{es-tu-dian-te}
\hyphenation{po-si-ti-vis-mo} \hyphenation{mo-de-li-za} \hyphenation{a-lea-to-ri-za-dos}
\hyphenation{me-jo-res} \hyphenation{con-fian-za} \hyphenation{es-pe-ra-da} \hyphenation{ge-ne-ral}
\hyphenation{ge-ne-ra-li-za-da}  \hyphenation{i-diomma}
\hyphenation{pro-yec-to}
\hyphenation{ac-ti-vi-da-des}
\hyphenation{es-tan-dar}


\pagestyle{fancyplain}

\renewcommand{\sectionmark}[1]
                {\markright{\thesection\ #1}}


\lhead[\fancyplain{}{\bfseries\thepage}]
      {\fancyplain{}{\bfseries\rightmark}}

\rhead[\fancyplain{}{\bfseries\leftmark}]
      {\fancyplain{}{\bfseries\thepage}}

\cfoot{}













\begin{document}
	
	
	
	

\title{Orientación Estadística\\Licenciatura en Matemática}


 \maketitle

 \newpage


\section{Asignatura Obligatoria}

\begin{itemize}
	\item \textbf{ Modelos de Regresión y Métodos Empíricos}
Regresión lineal y clasificación. Técnicas de remuestreo. Selección en modelos lineales y regularización: Ridge, Lasso, Elastic net, Grouped Lasso, Ensambles.    Métodos basados en Árboles. 

Bibliografía: \cite{friedman, james, trevor, efron}.
\end{itemize}




\section*{Asignaturas optativas}
 
	
	 
 

\begin{enumerate}



\item \textbf{Inferencia Estadística}.
Estimación puntual. Métodos de Estimación.
Evaluación de Estimadores.  Estimación por
Intervalos. Test de Hipótesis.

 

Referencias: \cite{bergero, bickel, degroot, lehmann,sen, rohatgi, scher}.



\item \textbf{Modelos Lineales}. Modelos de regresión. Análisis de la varianza. Distribución de formas cuadráticas y lineales. Estimación e inferencia para modelos lineales. 
 
 

 Referencias:  \cite{hock,  radrao, rawlyngs, scheffe, searle,  Venables, Montgonery}.
 
 
  

\item \textbf{Análisis Multivariado}. Formas lineales y transformaciones de matrices de datos normales. Estimación puntual. Test de hipótesis. Análisis de regresión multivariado. 
Análisis de componentes principales. Análisis factorial. Análisis de correlación canónica. Análisis discriminante. Análisis de la varianza multivariado. Análisis de clusters. Escalamiento multidimensional. Datos direccionales.

 
Referencias:  \cite{anderson, Everitt,  Hair, Jimenez2005,  Cuadras, mardia, Penia, johnson, Rencher}.







\item \textbf{Estadística computacional}. 
Optimización   continua y combinatoria. Algoritmos de ascensos por coordenadas. Simulated annealing. Algortimo EM. Simulación. Cadena de Markov Monte Carlo. Algoritmo Metropolis Hastings. Muestreo de Gibbs. Bootstrap. Bootstrap robusto.  


Referencias:    \cite{wrma, wrre, giho}




\item \textbf{Procesos Estocásticos}. Teorema de extensión de  medidas de Kolmogorov. Construcción de procesos a partir
de las distribuciones finito-dimensionales. Teorema de  clase $\pi-\lambda$. Esperanza condicional. Martingalas a tiempo
discreto. Desigualdades fundamentales. Teoremas de convergencia. Cadenas de Markov en espacio de estados discretos.
Clasificación de estados. Medidas invariantes. Teoría ergódica. Transformaciones que preservan medida. Teorema ergódico.

 

 Referencias: \cite{bremaud,ferrari, shir,   varadhan}.



\end{enumerate}

%\nocite{}
\bibliographystyle{plain}
\bibliography{probabilidad2}






\end{document}
