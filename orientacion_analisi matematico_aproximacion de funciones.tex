\documentclass[10pt,a4paper]{article}
\usepackage[utf8]{inputenc}
\usepackage{amsmath}
\usepackage{amsfonts}
\usepackage{amssymb}
\usepackage{enumerate}

\begin{document}

\begin{center}
 \emph{Análisis Matemático y Aproximación de Funciones}
\end{center}

\begin{enumerate}

\item\textbf{Introducción a la Teoría de Aproximación (2278):}
Existencia, unicidad y caracterización de mejores aproximantes desde subespacios en espacios normados. Aproximación por polinomios algebraicos.  Mejor aproximación en espacios clásicos de Lebesgue. Algoritmos. Desigualdades polinomiales. Orden de aproximación. 

\noindent \textit{Bibliografía sugerida}

\begin{enumerate}[]
\item	$[1]$ E.W. Cheney, Introduction to Approximation Theory, McGraw-Hill, New York, 1966. 
\item $[2]$ A. Iske, Approximation Theory and Algorithms for Data Analysis, Texts in Applied Mathematics Volume 68, Springer, Germany, 2018.
\item	$[3]$ G.G. Lorentz, Approximation of Functions, AMS Chelsea Publishing, New York, 1986. 
\item	$[4]$ A. Pinkus, On L1-Approximation, Volumen 93 de Cambridge tracts in mathematics/B.Bollobas, H. Halberstma \& C.T.C. Wall, Cambridge University Press, Cambridge, 1989.
\item	$[5]$ M.J. Powell, Approximation Theory and Methods, Cambridge University Press, Cambridge, 1981
\item	$[6]$ J.R. Rice, The approximation of functions: Vol 1, Linear Theory, Addison Wesley,  Reading, MA, 1964. 
\item	$[7]$ I. Singer, Best Approximation in Normed Linear Spaces by Elements of Linear Subspaces, Publishing House of the Academy of the Socialist Republic of Romania, Bucharest, 1970.
\end{enumerate}

\item\textbf{ Aproximación Simultánea en Espacios Normados (2269):} Existencia,  unicidad y caracterización de mejores aproximantes simultáneos desde subconjuntos
convexos.  Reducción  de  mejores aproximantes simultáneos a mejores aproximantes.   Puntos  extremales  en  espacios  productos. Caracterización  funcional de  mejores aproximantes  simultáneos  Aplicaciones  a  los espacios: $C[a,b]$, $L^1[0,1]$ y $L^p[0,1]$, $1 < p <\infty$.

\noindent \textit{Bibliografía sugerida}

\begin{enumerate}[]

\item $[1]$  A.R. Alimov, I.G. Tsar’kov. Chebyshev centres, Jung constants, and their applications, Russian Math. Surveys, 74 (5) (2019) 775–849

\item $[2]$	 A.S.B. Holland, S. Sahab. Some remarks on simultaneous approximation, Theory of Approx. with Applications, Academic Press, N. York, 1976.

\item $[3]$ 	Y. Karakus. On simultaneous approximation, Note di Matematica, 21 (1) (2002) 71-76.

\item $[4]$ P.K. Lin. Strongly unique best approximation in uniformly convex Banach spaces, J. Approx. Theory., 56 (1989) 101,107.

\item $[5]$.	A. Pinkus. Uniqueness in vector-valued approximation, J. Approx. Theory., 73 (1993) 17-92.

\item $[6]$.	J. Shi, R. Houtari. Simultaneous approximations from convex sets, Computers Math. Applic., 30 (3-6) (1995) 197-206.

\item $[7]$ M.L. Soriano Comino. Aproximación simultánea en espacios normados, Universidad de Extremadura, 1990.
\end{enumerate}

\item\textbf{Espacios de Funciones Invariantes por Reordenamiento:} Espacios de funciones de Banach. El espacio asociado. Espacios de funciones invariantes por reordenamiento. Funciones de distribución y reordenadas decrecientes. Espacios invariantes por reordenamiento. La función fundamental.  Espacios de Lorentz, $L_1+ L_\infty$ y $L_1 \cap L_\infty$. Índices de Boyd. Espacios de Orlicz-Lorentz. Teoremas de interpolación clásicos. El Teorema de Riesz-Thorin y el Teorema de Marcinkiewicz. Los espacios de Lorentz-Zygmund, $LlogL$ y $Lexp$.

\noindent \textit{Bibliografía sugerida}

\begin{enumerate}[]

\item $[1]$ C. Bennet, R. Sharpley, Interpolation of Operators, Academic Press, 1988.

\item $[2]$ M.J. Carro, J.A. Raposo, J. Soria, Recent Developments in the Theory of Lorentz Spaces and Weighted Inequalities, AMS, 2007.

\item $[3]$ J. Lindenstrauss, L. Tzafriri, Classical Banach Spaces II, springer, 2013.

\item $[4]$ S.G. Krein, Ju.I. Petunin, E.M. Semenov, Interpolation of linear operator, AMS, 1982.

\end{enumerate}

\item\textbf{Espacios de Orlicz e Interpolación:} Espacios modulares. Ejemplos. Espacios de Orlicz y clases de Orlicz. Separabilidad. Existencia y no existencia de funcionales lineales continuos. Función complementaria y norma de Orlicz. Forma general de funcionales lineales continuos. El producto de funciones y el Teorema de Landau. Indices es espacios de Orlicz. Espacios de Orlicz generados por funciones de Young. Teorema de interpolación no lineal de Orlicz. Interpolación en espacios de Orlicz. Espacios de Calderón-Lozanovskii e interpolación de operadores.

\noindent \textit{Bibliografía sugerida}

\begin{enumerate}[]

\item $[1]$  P. Harjulehto, P. Hästö, Orlicz Spaces and Generalized Orlicz Spaces, Springer, 2019.

\item $[2]$ V M Kokilashvili, M. Krbec, Weighted inequalities in Lorentz and Orlicz spaces, World Scientific, 1991.

\item $[3]$ M. A. Krasnosel'skii, Yz. B. Rtuickii, Convex Functions and Orlicz Spaces, Noordhoff, Ltd., 1961.

\item $[4]$ J. Lang, O.F. Memdez, Real-Variable Theory of Musielak-Orlicz Hardy Spaces, CRC Press, 2019.

\item $[5]$ L. Maligranda, Orlicz spaces and interpolation,  Universidade Estadual de Campinas, 1989.

\item $[6]$ M.M. Rao, Z.D. Ren, Theory of Orlicz spaces, Marcel Dekker, 1991.

\end{enumerate}

\item \textbf{Análisis Complejo:} Principio de modulo máximo. Teorema de
Rouche. Productos infinitos. Descomposición en fracciones simples,
Teorema de Mittag-Leffer. Funciones Armónicas. Función $\zeta$ de
Riemmann. Aplicaciones a la Teoría de Números. Continuación
Analítica. Superficies de Riemann. Bibliografía sugerida:
\cite{ahlfors,conway}

\item\textbf{Complementos de Análisis Real:} Medidas abstractas. Medidas de
Radon. Teorema de Radon-Nikodyn. Derivación de Medidas. Lemas de
cubrimiento. Diferenciación de la integral. Desigualdad maximal de
Hardy y Littlewood. Teorema de diferenciación de Lebesgue.
Aproximaciones de la identidad. Aplicaciones al problema de
Dirichlet al semiespacio. Dualidad de los espacios $L^p$.
Convergencia débil. Biblio\-grafía sugerida:
\cite{evansgariepy,favazo,rudin}.

\end{enumerate}

\end{document}